\begin{minipage}{.34 \textwidth}
    \setlength{\parindent}{2em}
    \hspace{0.5cm}Теорема Вейерштрасса спасает мно
    гие математические рассуждения, ко
    торые без нее оказались бы нестроги
    ми Сравним лва рассуждения.
    
    \hspace{0.5cm}\begin{large} П р и м е р \end{large}  \begin{Large} 1\end{Large}. Положим
    \begin{equation}
        x = \sqrt{2+\sqrt{2+\sqrt{2+ ...}}}
    \end{equation}
    (количество радикалов бесконечно).
    Заметим, что под первым радикалом стоит 2+$x$. Решая уравнение
    \begin{equation}
        x = \sqrt{2+x}
    \end{equation}
    получаем $x$=2
    
    \hspace{0.5cm}\begin{large} П р и м е р \end{large}  \begin{Large} 2\end{Large}. Положим
    \(x = 1 + 2 + 4 + 8 + 16 + 32 + ...\)
    (количество слагаемых бесконечно).
    Заметим, что
    
    \hspace{0.5cm}\(x = 1 + 2 + 4 + 8 + 16 + ... = \)
    
    \hspace{0.5cm}\( = 1 + 2(1 + 2 + 4 + 8 + ...) = \)
    
    \hspace{4.2cm}=\(1 + 2x\)
\end{minipage}
\begin{minipage}{.34\textwidth}
    \hspace{0.5cm}Второе рассуждение спасти та-
    ким путем невозможно (см. упр. 8).
    \[***\]
    \hspace{0.5cm}Вернемся к началу статьи. Так
    есть ли предел у рекордов? Казалось
    бы, все ясно: последователь ность
    рекордов, скажем, в беге на 100м,
    конечно, убывает (каждый следующий
    рекорд фиксирует меньшее время)
    и ограничена снизу (никто никогда
    не пробежит 100м за 0 секунд); зна-
    чит, по теореме Вейерштрасса она
    имеет предел.
    
    \hspace{0.5cm}На самом деле, в применении ма-
    тематического аппарата к реальной
    (в данном случае, к спортивной)
    жизни не все так просто, как это сы-
    глядит в предыдущем абзаце. Вот и
    у нас, конечно же, не все корректно.
    Что именно? Ожидаем ваших писем.
    \\
\end{minipage}